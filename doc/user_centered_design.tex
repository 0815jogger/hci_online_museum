
%%%%%%%%%%%%%%%%%%%%%%% file typeinst.tex %%%%%%%%%%%%%%%%%%%%%%%%%
%
% This is the LaTeX source for the instructions to authors using
% the LaTeX document class 'llncs.cls' for contributions to
% the Lecture Notes in Computer Sciences series.
% http://www.springer.com/lncs       Springer Heidelberg 2006/05/04
%
% It may be used as a template for your own input - copy it
% to a new file with a new name and use it as the basis
% for your article.
%
% NB: the document class 'llncs' has its own and detailed documentation, see
% ftp://ftp.springer.de/data/pubftp/pub/tex/latex/llncs/latex2e/llncsdoc.pdf
%
%%%%%%%%%%%%%%%%%%%%%%%%%%%%%%%%%%%%%%%%%%%%%%%%%%%%%%%%%%%%%%%%%%%


\documentclass[runningheads,a4paper]{llncs}

\usepackage{amssymb}
\setcounter{tocdepth}{3}
\usepackage{graphicx}
\usepackage[affil-it]{authblk}
\usepackage[utf8]{inputenc}
\usepackage{cite}
%\usepackage[numbers]{natbib}
\usepackage[english, main=ngerman]{babel}% deutsche Trennregeln
%\usepackage[T1]{fontenc}% wichtig für Trennung von Wörtern mit Umlauten
\usepackage{microtype}% verbesserter Randausgleich
\usepackage[hidelinks]{hyperref}
\usepackage{multirow}

\usepackage{url}
%\urldef{\mailsa}\path|{alfred.hofmann, ursula.barth, ingrid.haas, frank.holzwarth,|
%\urldef{\mailsb}\path|anna.kramer, leonie.kunz, christine.reiss, nicole.sator,|
%\urldef{\mailsc}\path|erika.siebert-cole, peter.strasser, lncs}@springer.com|    
\newcommand{\keywords}[1]{\par\addvspace\baselineskip
\noindent\keywordname\enspace\ignorespaces#1}

\begin{document}

\mainmatter  % start of an individual contribution

% first the title is needed

\title{Virtuelles Museum\\Analyse\\ Prototyping\\ Evaluation}

% a short form should be given in case it is too long for the running head
\titlerunning{Virtuelles Museum}

% the name(s) of the author(s) follow(s) next
%
% NB: Chinese authors should write their first names(s) in front of
% their surnames. This ensures that the names appear correctly in
% the running heads and the author index.
%

\author{
		Adrian Franken (Matr.nr.:s0538115),\\
		Chris-Andr\'{e} Posselt (Matr.nr.:s),\\
		Elsa Buchholz (Matr.nr.:s0544180),\\ 
		Igor Olegovich Turanin (Matr.nr.:s)
	   }

%
\authorrunning{Virtuelles Museum}
% (feature abused for this document to repeat the title also on left hand pages)

% the affiliations are given next; don't give your e-mail address
% unless you accept that it will be published
%\institute{Hochschule für Technik und Wirtschaft Berlin\\}
\institute{
		Hoschschule für Technik und Wirtschaft Berlin\\
}
\affil{	
		Fachbereich: Informatik, Kommunikation und Wirtschaft\\
		Studiengang: Angewandte Informatik\\
		Seminar: Human-Computer Interaction\\
		Seminarverantwortlicher: Prof. Dr.-Ing. Johann Habakuk Israel \\
}


%
% NB: a more complex sample for affiliations and the mapping to the
% corresponding authors can be found in the file "llncs.dem"
% (search for the string "\mainmatter" where a contribution starts).
% "llncs.dem" accompanies the document class "llncs.cls".
%

%\toctitle{Lecture Notes in Computer Science}
%\tocauthor{Authors' Instructions}

{\def\addcontentsline#1#2#3{}\maketitle}
\renewcommand{\contentsname}{Inhaltsverzeichnis}
%\maketitle
%\thispagestyle{empty}
\newpage
%\thispagestyle{empty}
\tableofcontents
\newpage
\thispagestyle{empty}
\listoffigures
\newpage
\thispagestyle{empty}
\listoftables
\newpage
\setcounter{page}{1}

\selectlanguage{english}
\begin{abstract}
	

\keywords{}
\end{abstract}
\selectlanguage{ngerman}

\section{Einleitung}
\section{Anforderungsanalyse}
\subsection{Anwendungsfälle und Nutzungskontext} \label{sect:anwendung}
Laut Miedler 2010 ist ein Museum eine Einrichtung, die zu Bildungs- und Forschungszwecken Zeugnisse von Umwelt und Menschen, der Öffentlichkeit zur Verfügung stellt und aufbewahrt. Die Zeugnisse von Umwelt und Menschen können sehr vielfältig sein. Dabei kann es sich um Bauarten wie Schlösser und Burgen handeln, als auch um Kunstobjekte aus der Malerei oder Fotografie sowie naturwissenschaftliche oder technische Erzeugnisse. Somit lassen sich Museen in verschiedene Arten einteilen, wie beispielsweise in Burg- und Schlossmuseum, Kunstmuseum, technisches oder naturkundliches Museum. Nicht nur der Bildungs- und Forschungszweck eines Museums soll erfüllt werden, sondern das Erleben der Zeugnisse soll innerhalb einer Ausstellung erreicht werden. Ausstellungen können dabei dauerhaft oder wechselnd kuratiert werden, wobei sie je nach Art des Museums an einem bestimmten Ort gebunden sind. Aufgrund der Vielzahl von Museumsarten und dem Zweck des Bildungs- und Forschungsauftrages bzw. Erlebnisses, besitzen Museen eine weitgefasste Zielgruppe. Daraus ergeben sich Interessengruppen, die sich aus der jeweiligen Art des Museums oder aus verschiedenen Berufsgruppen wie beispielsweise Lehrer, Schüler, Studenten oder Mitarbeiter des Museums wie Kuratoren oder Archivare  ergeben\cite[S. 29]{Miedler.2010}.\\ 

In der folgenden Tabelle werden die Stakeholder des Systems zusammengefasst. Sie beschreibt welches Interesse und welchen Einfluss die Stakeholder auf das System haben.\\

\begin{table}
	\begin{tabular}{|c|c|c|}\hline
	Stakeholder & Interesse & Einfluss\\ \hline
	\multirow{2}*{aktive Museumsgänger} & neue Möglichkeiten &\\  & zur Erkundung der Exponate ermöglichen & sehr hoch \\ \hline
	
	nicht aktive Museumsgänger & Interesse am Museumsbesuch wecken  & hoch\\ \hline
	
	\multirow{2}*{Archivar des Museums}& Erfassung der Exponate &\\ & innerhalb einer Datenbank & sehr gering\\ \hline
	
	\multirow{2}*{Kurator des Museums} & Integration von Technologien&\\& innerhalb einer Ausstellung & hoch\\ \hline
	pädagogische Kräfte  & Lehreiche Informationen erhalten& sehr hoch\\ \hline
	Kinder  & spielerischer Umgang mit den Exponaten & sehr hoch\\ \hline
	\end{tabular}
\end{table}

Daraus ergeben sich folgende Ziele:
\begin{itemize}
	\item Aktive Museumsgänger sollen weiterhin motiviert werden in das Museum zu gehen und durch neue Technologien mit den Exponaten interagieren können. Die Interaktion soll dem Kenner des Museums ermöglichen, neue Informationen bzw. einen neuen Zugang zu diesen zu gewinnen.
	\item Nicht aktive Museumsgänger sollen durch neue Technologien einen Zugang zum Museum erhalten, wobei das Interesse für die Exponate geweckt werden soll.
	\item Mitarbeiter des Museums sollen von den Technologien profitieren, indem sie Exponate besser Verwalten können bzw. die Ausstellung mit einem höheren Gehalt an Interaktionen konzipieren können.
	\item Pädagogische Kräfte sollen Wissen mit Hilfe des Museums verständlicher vermitteln können.
	\item Kinder sollen spielerisch Lernen können und mit Freude das Museum als Lernraum entdecken.\\
	
\end{itemize}

Das virtuelle Museum kann die Grenzen des Ortsbezuges aufheben. Beispielsweise kann ein virtuelles Museum über das Internet erreichbar sein, sodass die Ausstellungsstücke unabhängig von einem bestimmten Ort zugänglich sind. Aufgrund des Standes der Technik ist es möglich das Erleben der Ausstellungstücke durch das Interagieren und Herstellen von Zusammenhängen im virtuellen Raum greifbar zu machen. Durch das Bereitstellen von Informationen im virtuellen Raum auf individuelle Art und Weise kann vor allem dem Bildungsanspruch entsprochen werden und somit im besonderen der Zielgruppe Lehrer und Schüler entsprochen werden. Mitarbeiter eines Museums wie beispielsweise Kuratoren oder Archivare könnten vom virtuellen Museum profitieren innerhalb einer dreidimensionalen Datenbank als Überblick zur Verwaltung und zum Austausch von Ausstellungsstücken oder beim virtuellen zusammenstellen von Ausstellungen. Im folgenden Abschnitt zeigt der Stand der Technik weiterführende Möglichkeiten wie ein virtuelles Museum verstanden werden kann.\\

\subsection{Stand der Technik} \label{sect:standtechnik}

Eine Kategorie wie ein virtuelles Museum verstanden werden kann, ist das Online-Museum. Darunter können Museen fallen, die im Internet virtuell, begehbar und interaktiv nutzbar sind. Ein Beispiel dafür ist das Google Art Projekt \cite{GoogleCultureInstitut.2011}. Es ermöglicht Ausstellungshäuser weltweit zu erkunden. Auf dieser Seite können sich Nutzer durch die Bestände der Museen klicken und eigene Galerien anlegen, sowie virtuelle Rundgänge durch die Museen machen. Ziel eines Online-Museums ist es die Nutzer in das entsprechende Museum zu locken.\\ 

Die zweite mögliche Kategorie virtuelle Museen einzuordnen ist das VR-Museum. Dabei wird die VR-Technologie verwendet, um Nutzer mittels einer App und einer VR-Brille einen Museumsbesuch zu ermöglichen. Diese Variante ist Ort- und Zeitunabhängig. Das Museum für Naturkunde in Berlin verwendet die Technologie, um einen Dinosaurier zum Leben zu erwecken. Dabei kann der Besucher sich vor Ort eine VR-Brille ausleihen oder via App Ortsunabhängig sich den Dinosaurier ansehen. In einer dritten Kategorie können mittels AR-Technologie sich auf einem Tablet zusätzliche Informationen zu Ausstellungsstücken angezeigt werden, sowie es das Bayerische Nationalmuseum in München macht \cite{GoetheInstitute.V..2017}.\\

Im Bereich der 3D-Druck Technologie können Ausstellungsstücke als 3D-Objekt ausgedruckt werden. Das Projekt Museum in a Box verleiht gedruckte 3D-Objekte aus einem Museum in einer kleinen Box. Die Ausstellungsstücke in Miniformat kommunizieren über das Internet, um Informationen zu einem bestimmten Objekt abzurufen und sprachlich wiederzugeben. Somit werden die Ausstellungsstücke anfassbar und können in einen bestimmten Kontext gestellt werden \cite{GeorgeOates.2015}.\\

Im Bereich der Tangible Interfaces Technologie kann die 3D-Druck Technologie verwendet werden, um 3D-Ausdrucke von Ausstellungsstücken als Interface zu nutzen. Die 3D-Objekte sind mit Sensoren und Elektronik ausgestattet, sodass mit Ausstellungsstücken interagiert werden kann \cite{Capurro.2015}.\\ 

Zusammenfassend lässt sich sagen, dass die Technologien 3D-Druck, Augmented Reality (AR) und Virtuell Reality (VR) es ermöglichen ein virtuelles Museum auf unterschiedlichste Art und Weise entstehen zu lassen. Dabei können die Technologien ein Museum virtuell ergänzen oder es gänzlich ersetzen. Die Interaktion innerhalb eines virtuellen Museums ist mit Tangible User Interfaces denkbar, bei dem dreidimensionale Objekte mit Hilfe von einem anfassbaren Objekt die virtuellen Ausstellungsstücke bewegt und gesteuert werden. Der 3D-Druck kann Ausstellungsstücke in Miniformat haptisch erlebbar und zu einem Sammelobjekt, das mit einem Informationsgewinn verknüpft wird, gemacht werden. Die Technologie AR kann ein Museum ergänzen, indem Ausstellungstücke in einen virtuellen Kontext gesetzt werden oder mit Informationen versehen werden, die der Nutzer sich individuell anpassen kann. Der Nutzer kann entscheiden, wie die Information ausgeben wird. Vorstellbar ist, dass die Sprache oder Textform, die Tiefe der Information und die Geschwindigkeit der Darstellung gewählt werden können. Im folgenden Abschnitt werden die Ziele und Ergebnisse der durchgeführten Fokusgruppe zum Thema virtuelle Museen vorgestellt.\\

\subsection{Fokusgruppe}
Im Rahmen der Fokusgruppe wurde untersucht, wie die Interaktion mit den Ausstellungsstücken in einem virtuellen Museum gestaltet werden kann. Dafür wurde zunächst erörtert wie ein klassischer Museumsbesuch und die Interaktion mit den Ausstellungsstücken aussieht. Im zweiten Schritt wurde herausgearbeitet wie in einem virtuellen Museum die Interaktion mit den Ausstellungsstücken aussehen könnte. Ziel war es herauszufinden wie sich die Nutzer ein interaktives Museum vorstellen, welche Technologien dafür verwendet werden sollten und wie die Interaktion mit Hilfe entsprechender Technologien auf ein virtuelles Museum anwendbar ist.\\

Insgesamt waren vier Teilnehmer an der Fokusgruppe beteiligt. Jeder der Teilnehmer war schon einmal in einem Museum, wobei drei der Teilnehmer mindestens zweimal im Jahr ins Museum gehen. Dabei handelt es sich vor allem um Kunstmuseen oder Fotoausstellungen.\\

Die Fokusgruppe wurde von einem Komoderator und einem Moderator geführt und von zwei Protokollanten schriftlich festgehalten. Nach einer inhaltlichen Einleitung haben sich die Teilnehmer vorgestellt, indem sie die Fragen beantworten sollten was der Anlass eines Museumsbesuch ist und welche Erfahrungen sie bisher während eines Museumsbesuchs gemacht haben. Nach der Vorstellungsrunde wurden zur Weiterentwicklung des Gesprächs folgende offene Fragen beantwortet, die sich an einem halboffen Leitfaden orientierten:
\begin{itemize}
	\item Was erwarte ich von einem guten Museumsbesuch?
	\item Wie sollen die Ausstellungsstücke präsentiert werden?
	\item Wie sieht ein Museumsbesuch aus? Gibt es Vor- oder Nachbereitungen?
	\item Wie agiert ihr mit den Ausstellungsstücken?
\end{itemize}

Als gute Museen wurden das Mathematikum Gießen, das Naturkundemuseum Berlin und das jüdische Museum Berlin benannt. Das Mathematikum begeistert, da in Form von interaktiven Experimenten Mathematik erlebt werden kann, wohingegen das Naturkundemuseum vor allem durch seine Ausstellungsstücke wie den Dinosaurier in Lebensgröße besticht. Beeindruckend ist das Jüdische Museum, weil es mit begehbaren Installationen die Ausstellungsstücke erlebbar macht.\\

Grundsätzlich wird ein Museum auf Reisen bzw. im Urlaub besucht ohne Vor- oder Nachbereitungen, sondern eher spontan. Dabei wird Kunst als interessant bezeichnet und am ehesten werden Fotoausstellungen besucht. Innerhalb des Museums wird sich treiben gelassen und durch die Ausstellungsräume geschlendert. Dabei werden die Tafeln für den Informationsgewinn gelesen. Ein Museumsbesuch sollte Lehrreich sein, wobei Interaktionen nicht wichtig sind. Weiterhin wird das Museum gerne genutzt, um  Freunde zu treffen und gemeinsam über die Ausstellungsstücke zu sprechen. Wenn es was zum Anfassen gibt, wird das gerne ausprobiert. Meist wird mit den Ausstellungsstücken durch Beobachtungen interagiert.\\

Nach der Diskussion wurde das Video Museum in a Box gezeigt, indem die Organisation ihr Konzept vom Museum in der Box darstellt. Das Video dient als Überleitung zu virtuellen Museen als eine neue Form der Museen mit veränderten Möglichkeiten zur Interaktion mit Ausstellungsstücken. Im folgenden wurde der Frage nachgegangen wie sie sich ein virtuelles Museum vorstellen und wie darin mit Ausstellungsstücken interagiert werden kann.\\

In Form einer Diskussion haben die Teilnehmer die Kategorien eines virtuellen Museum aus dem Abschnitt \nameref{sect:standtechnik} herausgearbeitet und bewertet. Die Tabelle fasst die Ergebnisse dieser Diskussion zusammen.\\

\begin{table}
\begin{tabular}{|c|c|}\hline
	Kategorie							& Beurteilung\\ 
										\hline
	\multirow{2}*{Online-Museum}		& negativ bewertet,\\
						  				& da viel verloren geht aus dem Museum \\  \hline 
	\multirow{5}*{Virtual Reality}		& innerhalb eines Museums negativ, weil man\\
	                                    & das zu Haus machen kann\\
	                      				& Technik selber als positiv bewertet, weil\\
	                      				& damit unmögliches möglich gemacht werden kann\\
	                     				& unerreichbare Orte sind damit erreichbar\\ \hline
	\multirow{3}*{Augmented Reality}	& positiv in einem Museum,\\ 
				                    	& um über Gesten, Schieben und Ziehen\\
	                     				& mit Ausstellungsstück zu agieren\\
	                     				 \hline
	\multirow{4}*{3D-Druck}				& positiv bewertet\\
										& Figuren als Teaser, um Ausstellungstücke\\
										& im Museum in echt zu sehen\\
										& zusammenstellen von eigener Ausstellung\\ \hline
	\multirow{1}*{Tangible Interface}	& nicht benannt\\ 

										\hline

\end{tabular}\\
\caption{Bewertung der Kategorien eines virtuellen Museums als Interaktionsmöglichkeiten}
\end{table}

Während der Diskussion sind viele Ideen entstanden, wie die Technologien verwendet werden können. Dabei stand vor allem die spielerische Auseinandersetzung mit den Ausstellungsstücken eine Rolle. Es entstanden Ideen wie beispielsweise das eigene Zusammenstellen und ausdrucken einer eigenen Ausstellung, das Nachstellen und Erleben von Unmöglichen Situationen wie der Reise zu Mond oder 1000 Meter unter dem Meer oder das spielerische nachzeichnen der Mona Lisa, sowie das Umschmeißen im Duell mit Freunden einer Ming-Vase.\\

Zusammenfassend ergeben sich aus den definierten Zielen aus dem Kapitel \nameref{sect:anwendung} und folgenden Zitaten die Funktionen zur Entwicklung des Systems. 
Die beiden Zitate \glqq Webseiten sind doof, da geht viel zu viel verloren vom Museum.\grqq und \glqq Homepage auf gar keinen Fall! Eine Online-Galerie ist doof und es geht viel verloren.\grqq haben uns zu der Entscheidung gebracht, das System nicht als Webseite umzusetzen. Die ursprüngliche Idee ein interaktives Onlinemuseum in Form einer Webseite mit 3D-Objekten zu entwickeln, wird zu Gunsten der Idee interaktiv in einem Museum zu handeln verworfen. Die Diskussion ergab, dass Interaktionen in einem virtuellen Museum weniger über Webseiten zu realisieren sind, als über Technologien wie AR, VR oder 3D-Druck. Das Zitat
\glqq Die Situation erlebbar zu machen ist wichtig in einem virtuellen Museum. Orte wo man nicht hin kann.\grqq 
%Begriff virtuelles Museum gewählt um offener über Anforderungen sprechen zu können. Interaktion hauptaugenmerk.
zeigt den Wunsch VR-Technologien für ein virtuelles Museum anzuwenden, indem interaktiv agiert werden kann. Wohingegen die Zitate
\glqq Spannend ist es, wenn es zusätzlich zum Museum angeboten wird.\grqq \glqq Nebeninformationen sind super!\grqq den Wunsch nach AR-Technologie äußern. 
Daraus lassen sich für die Entwicklung des Systems die in der Tabelle aufgeführten Funktionen und Anforderungen ableiten.

\begin{table}
\begin{tabular}{|c|c|}\hline
	Funktion									& 	Anforderung\\ \hline
	
\multirow{7}*{Präsentation von Informationen}	& Das System muss dem Nutzer \\
												& die Möglichkeit bieten\\
 												& Informationen zum Exponat\\
												& 1. anzuzeigen\\ 
												& 2. zu sortieren\\
												& 3. zu filtern\\
												& 4. zu vertiefen.\\\hline
											
\multirow{7}*{Interaktion mit Exponaten} 		& Das System soll dem Nutzer\\
												& die Möglichkeit bieten\\
												& das Exponat\\
												& 1. zu erkunden\\
												& 2. zu gestalten\\
												& 3. in einen Kontext zu setzten\\
												& 4. spielerisch zu erfassen.\\ \hline

\multirow{4}*{Individualisierung}				& Das System soll dem Nutzer\\
												& die Möglichkeit bieten\\
												& 1. eigenen Interessen zu folgen\\
												& 2. eigene Ausstellungen zu gestalten\\ \hline
												
\multirow{3}*{Lerneigenschaft}					& Das System soll dem Nutzer\\
												& die Möglichkeit bieten\\
												& etwas über die Exponate zu lernen.\\ \hline
												
\multirow{4}*{Erreichbarkeit von Orten}			& Das System soll dem Nutzer\\
												& die Möglichkeit bieten\\
												& 1. andere Welten zu erkunden\\
												& 2. an unerreichbare Orte zu gelangen.\\ \hline
												
\multirow{4}*{Spielcharakter}					& Das System soll dem Nutzer\\
												& die Möglichkeit bieten\\
												& 1. 3D-Objekte von den Exponaten zu sammeln\\
												& 2. ein gemeinsames Erlebnis mit Freunden zu bieten.\\ \hline							
\end{tabular}
\end{table}


%Das System muss Fähig sein das Exponat zu repräsentieren.

%Das System muss Fähig sein mit dem Nutzer zu agieren.

%Das System soll Fähig sein dem Nutzer Wissen zu vermitteln.

%Das System muss eine motivierende Wirkung auf die Nutzer für die Exponate ausüben.

%Das System soll spielerisch angewandt werden.

Im nächsten Abschnitt wird beschrieben, wie die gewonnen Erkenntnisse über die Anforderungen eines virtuellen Museums in Form eines Papierprototypen umgesetzt werden.\\

\section{Prototyping: low fidelity}
\subsection{Papierprototyp}
\subsection{Designentscheidung}
\subsection{heuristische Analyse}

\section{Prototyping: high fidelity}
\subsection{Umsetzung des Papierprototypen}

\section{Zusammenfassung}


\newpage
\bibliographystyle{splncs03} 
\bibliography{bib}
\end{document}
