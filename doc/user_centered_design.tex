
%%%%%%%%%%%%%%%%%%%%%%% file typeinst.tex %%%%%%%%%%%%%%%%%%%%%%%%%
%
% This is the LaTeX source for the instructions to authors using
% the LaTeX document class 'llncs.cls' for contributions to
% the Lecture Notes in Computer Sciences series.
% http://www.springer.com/lncs       Springer Heidelberg 2006/05/04
%
% It may be used as a template for your own input - copy it
% to a new file with a new name and use it as the basis
% for your article.
%
% NB: the document class 'llncs' has its own and detailed documentation, see
% ftp://ftp.springer.de/data/pubftp/pub/tex/latex/llncs/latex2e/llncsdoc.pdf
%
%%%%%%%%%%%%%%%%%%%%%%%%%%%%%%%%%%%%%%%%%%%%%%%%%%%%%%%%%%%%%%%%%%%


\documentclass[runningheads,a4paper]{llncs}

\usepackage{amssymb}
\setcounter{tocdepth}{3}
\usepackage{graphicx}
\usepackage[affil-it]{authblk}
\usepackage[utf8]{inputenc}
\usepackage{cite}
%\usepackage[numbers]{natbib}
\usepackage[english, ngerman]{babel}% deutsche Trennregeln
%\usepackage[T1]{fontenc}% wichtig für Trennung von Wörtern mit Umlauten
\usepackage{microtype}% verbesserter Randausgleich

\usepackage{url}
%\urldef{\mailsa}\path|{alfred.hofmann, ursula.barth, ingrid.haas, frank.holzwarth,|
%\urldef{\mailsb}\path|anna.kramer, leonie.kunz, christine.reiss, nicole.sator,|
%\urldef{\mailsc}\path|erika.siebert-cole, peter.strasser, lncs}@springer.com|    
\newcommand{\keywords}[1]{\par\addvspace\baselineskip
\noindent\keywordname\enspace\ignorespaces#1}

\begin{document}

\mainmatter  % start of an individual contribution

% first the title is needed

\title{Interaktives Onlinemuseum\\Analyse\\ Prototyping\\ Evaluation}

% a short form should be given in case it is too long for the running head
\titlerunning{Interaktives Onlinemuseum}

% the name(s) of the author(s) follow(s) next
%
% NB: Chinese authors should write their first names(s) in front of
% their surnames. This ensures that the names appear correctly in
% the running heads and the author index.
%

\author{
		Adrian Franken (Matr.nr.:s),\\
		Chris-Andr\'{e} Posselt (Matr.nr.:s),\\
		Elsa Buchholz (Matr.nr.:s0544180),\\ 
		Igor Olegovich Turanin (Matr.nr.:s)
	   }

%
\authorrunning{Interaktives Onlinemuseum}
% (feature abused for this document to repeat the title also on left hand pages)

% the affiliations are given next; don't give your e-mail address
% unless you accept that it will be published
%\institute{Hochschule für Technik und Wirtschaft Berlin\\}
\institute{
		Hoschschule für Technik und Wirtschaft Berlin\\
}
\affil{	
		Fachbereich: Informatik, Kommunikation und Wirtschaft\\
		Studiengang: Angewandte Informatik\\
		Seminar: Human-Computer Interaction\\
		Seminarverantwortlicher: Prof. Dr.-Ing. Johann Habakuk Israel \\
}


%
% NB: a more complex sample for affiliations and the mapping to the
% corresponding authors can be found in the file "llncs.dem"
% (search for the string "\mainmatter" where a contribution starts).
% "llncs.dem" accompanies the document class "llncs.cls".
%

%\toctitle{Lecture Notes in Computer Science}
%\tocauthor{Authors' Instructions}

{\def\addcontentsline#1#2#3{}\maketitle}
\renewcommand{\contentsname}{Inhaltsverzeichnis}
%\maketitle
%\thispagestyle{empty}
\newpage
%\thispagestyle{empty}
\tableofcontents
\newpage
\thispagestyle{empty}
\listoffigures
\newpage
\setcounter{page}{1}

\selectlanguage{english}
\begin{abstract}
	

\keywords{}
\end{abstract}
\selectlanguage{ngerman}

\section{Einleitung}
\section{Anforderungsanalyse}
\subsection{Anwendungsfälle und Nutzungskontext}
\subsection{Stand der Technik}
\subsection{Fokusgruppe}


\section{Prototyping: low fidelity}
\subsection{Papierprototyp}
\subsection{Designentscheidung}
\subsection{heuristische Analyse}

\section{Prototyping: high fidelity}
\subsection{Umsetzung des Papierprototypen}

\section{Zusammenfassung}


\newpage
\bibliographystyle{splncs03} 
\bibliography{bib}
\end{document}
